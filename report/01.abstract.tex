\begin{abstract}
  Artificial Intelligence (AI) is a hot topic, especially recently.
	This uptick in public interest in AI is driven by the improved capabilities of
  deep learning architectures such transformers.
	The emergent capabilities of such models, including GPT~\cite{radford2019language},
  has immense utility in other fields.
  So why has this dawn of AI has been met with conflicted sentiments?
	\\

  The very fact that recent neural network--based models have immense capabilities
	leaves professionals anxious about their jobs and creative products.
  The ethics of deploying AI models in critical, user-facing applications where
	errorneous predictions have real-world consequences is also a concern.
	In fact, a faction of industry leaders has even called for AI experiments to be paused
	until proper regulations are in place because of the risk of runaway AI.
	\\

  There is also a lingering question around the ownership of work generated by AI models.
	When a model's output is clearly derived from a copyrighted artifact,
	who owns the output?
	The original creator of the imitated artifact may not want to be associated
	with the new imitation, since it may have inferior quality or even be offensive.
	The same concern applies to the creators of the AI models themselves.
	At the same time, imitations of copyrighted works \emph{without permission}
	have their own legal implications.
	\\

  This research project studies the societal attitudes toward AI, both currently
	and how they have evolved over the years,
	as a way to understand how different events have shaped the public's perception of AI.
  We use topic modeling, sentiment analysis, and procrustes analysis
	to analyze relationships across time periods and extract insight
	into the changing story of artificial intelligence.
\end{abstract}

\keywords{AI \and Machine Learning \and Ethics}
