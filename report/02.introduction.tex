\section{Introduction}~\label{sec:introduction}
We sought to find a \emph{representative} yet \emph{accurate} sample of the public's opinion on AI.
We considered multiple potential sources of data, and their tradeoffs:
\begin{enumroman}
  \item \emph{Research papers} are perhaps the most factually accurate
    and scientifically substantial, but they
    tend to dive into detailed exposition of novel
    model architectures and their results, and can be disconnected
    from the public's opinion.
  \item \emph{Social media posts} are often either too short to offer a nuanced
    opinion, and are not fact-checked, so they can convey misleading information.
  \item \emph{News articles} can be sensationalized and biased, but they
    are often fact-checked and offer a more nuanced opinion than social media,
    while still being in touch with public opinion. 
\end{enumroman}

We decided to use news articles as our target data source.

We then considered potential forms of analysis to use as a lens to study the data.
\begin{enumroman}
  \item \emph{Topic modeling} can be used to identify the most common topics
    in the data, and how they change over time. This can be insightful in
    identifying when certain issues---such as \emph{ethics}
    and \emph{singularity}---became more or less emphasized
    in societal conversations around AI.
  \item \emph{Sentiment analysis} can be used to identify attitudes
    toward AI and how these attitudes change over time.
    This can be insightful in identifying when the public's attitude toward AI became more positive
    or negative.
  \item \emph{Procrustes analysis} can be used to identify how the conversations
    around AI in specific time periods are similar or different from each other.
    This can be insightful in identifying periods of significant shifts
    in how people talk about AI.
\end{enumroman}
